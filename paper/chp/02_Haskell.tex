\chapter{Haskell}

Haskell je funkcijski programski jezik sa mnogo značajki kojima ga se može opisati. Neke od najvažnijih značajki su čistoća, referencijalna transparentnost, lijenost, nepromjenjivost podataka i jak tipski sustav.

\section{Čistoća}

Čistoća je svojstvo jezika koje kaže da funkcija \textbf{uvijek} mora vraćati vrijednost te da osim vraćanja te vrijednosti ne smije imati nikakvih sporednih efekata, kao na primjer ispis na ekran, pisanje u bazu podataka ili lansiranje nuklearnih raketa.

Ovo svojstvo je izrazito korisno, pogotovo kod konkurentnog koda gdje postoji mogućnost da koristimo neku funkciju koja vraća određenu vrijednost ali u pozadini mijenja neku varijablu koje mi nismo svjesni što može utjecati na rezultat neke druge funkcije koja također čita i piše u tu istu varijablu.

Koliko god težili korištenju čistih funkcija na kraju je ipak potrebno pisati u bazu podataka ili mijenjati neko stanje. U tu svrhu koriste se monade. One su vrlo elegantno rješenje za ovaj problem te ujedno čuvaju čistoću jezika bez da moramo žrtvovat takve \textit{sporedne} akcije.

Razlika je u tome što monade zahtijevaju da su sve takve sporedne akcije eksplicitne, odnosno da su vidljive iz samog povratnog tipa podatka. Dok u nekom prljavom jeziku možemo imati neku funkciju koja ima tip povratne vrijednosti \texttt{Int} ali ujedno mijenja neku globalnu varijablu \texttt{String x} u Haskellu takva funkcija mora eksplicitno reći da mutira neko stanje, tako da bi u tom slučaju tip povratne vrijednosti bio \texttt{State String Int} gdje \texttt{State} jasno daje do znanja da funkcija utječe na neko \textit{stanje} koje je tipa \texttt{String} te ima povratnu vrijednost tipa \texttt{Int}. Također, prilikom pozivanja takve funkcije moramo predati početno stanje kao jedan od argumenata.

Ovakvim pristupom dali smo puno više korisnih informacija programeru kao i kompajleru koji na osnovu toga mogu donositi puno bolje zaključke o dijelovima koda i njihovom međudjelovanju.

\section{Referencijalna transparentnost}

Referencijalna transparentnost nam govori da čiste funkcije uvijek vraćaju istu vrijednost za iste argumente tako da ukoliko imamo funkcije \texttt{f(a, b) = a + b} i \texttt{g(a, b) = f(a,b) + f(b,a)} možemo slobodno zamijeniti \texttt{f(a,b)} u funkciji \texttt{g} sa \texttt{a + b + a + b} što kompajler može dalje optimizirati i pretvoriti u \texttt{2*a + 2*b}. Dakle, sigurni smo da će ponašanje funkcije ostati isto, samo će se razlikovati u performansama.

\section{Lijenost}

Radi referencijalne transparentnosti, funkcije mogu biti izvršene u bilo kojem trenutku i uvijek dati isti rezultat. Iz tog razloga je moguće odgoditi evaluaciju izraza za kasnije što nam otvara mogućnost rada sa vrlo apstraktnim elementima kao što su beskonačne liste ili \textit{bottom} vrijednosti.

\section{Nepromjenjivost podataka}

U Haskellu ne postoje varijable. Jednom kada smo definirali neku vrijednost ona ostaje zauvijek ne promijenjena. Umjesto mutiranja originalne vrijednosti ili strukture radi se kopija.

\section{Jak tipski sustav}

Haskell ima vrlo jak i izrazit tipski sustav. Postoje razne stvari koje nam omogućava, poput programiranja na razini tipa, automatsko zaključivanje tipova ili čak mogućnost automatske implementacije funkcije prema njenom tipu.

Korisno je pogledati jedan primjer te ga usporediti sa Javom. Usporedbu ćemo provesti na funkciji \texttt{misterij}.

\loadcode{haskell}{Haskell misterij tipski potpis}{haskell_misterij}{paper/cde/misterij_haskell.hs}

\loadcode{java}{Java misterij tipski potpis}{java_misterij}{paper/cde/misterij_java.java}

Ovdje su dani samo tipski potpisi. Ako se zapitamo kako bi mogla izgledati implementacija primijetit ćemo da funkcija vraća isti tip podatka koji i prima. Kada shvatimo da ne znamo ništa tom tipu jedina moguća implementacija je da vratimo istu vrijednost koju smo i primili, dakle \texttt{misterij} je zapravo funkcija identiteta.

Na žalost, iako to logika nalaže, istu stvar ne možemo tvrditi i za moguću implementaciju u Javi. U Javi \texttt{misterij} može vratiti \texttt{null}, saznati klasu i napraviti novu instancu, usporediti pokazivač i sl. Drugim riječima, tipovi u Haskellu \textbf{ne lažu}.