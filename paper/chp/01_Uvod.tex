\chapter{Uvod}

Često se dešava da ne znamo što bi htjeli za ručak. U dućanu obično bivamo inspirirani na licu mjesta pa znamo neplanirano kupiti i više namirnica no što nam treba. Takav pristup obično vodi do određene količine sastojaka koji ostaju neiskorišteni te propadaju u frižideru i na kraju budu bačeni.

Pošto većina ljudi ne želi svoje vrijeme trošiti na planiranje optimalnog ručka i liste za kupovinu kao ni na smišljanje jelovnika, ideja je bila napraviti aplikaciju koja će to odraditi umjesto njih.

Osim same aplikacije, cilj ovog projekta je istražiti primjenu čistog funkcijskog programskog jezika Haskell kao bolje alternative za razvoj softvera.

Ideja je proizašla iz nezadovoljstva nakon višegodišnjeg korištenja objektno orijentiranih programskih jezika te pojavom elemenata funkcijskog programiranja u popularnim programskim jezicima kao što su C\# i Java.

Današnje interakcije ljudi i različitih područja ljudskog djelovanja postaju sve složenije te važnost informacijske tehnologije kao glavnog vezivnog tkiva i kanala za razmjenu informacija postaje sve veća.

Iz tog razloga pojavljuju se sve složeniji sustavi i primjene programskih rješenja nad kojima se mora brzo iterirati uz maksimalnu sigurnost i kvalitetu. Takva dinamika zahtjeva efektivne alate za upravljanje kompleksnošću i tu se objektno orijentirani pristup pokazuje sve manje poželjnim izborom, dok se funkcijski pristup čini sve pogodnijim zbog svojih urođenih svojstava koja će biti izložena u nastavku.

% \Ac{url}
% \Ac{html}

% \begin{figure}
%     \centering
%     \includegraphics{haskell_logo}
%     \caption{Haskell logo}
%     \label{lbl:haskell_logo}
% \end{figure}

% \begin{table}
%     \centering
%     \begin{tabular}{||c c c c||} 
%         \hline
%         Col1 & Col2 & Col2 & Col3 \\ [0.5ex] 
%         \hline\hline
%         1 & 6 & 87837 & 787 \\ 
%         2 & 7 & 78 & 5415 \\
%         3 & 545 & 778 & 7507 \\
%         4 & 545 & 18744 & 7560 \\
%         5 & 88 & 788 & 6344 \\ [1ex] 
%         \hline
%     \end{tabular}
%     \caption{Table to test captions and labels}
%     \label{table:1}
% \end{table}

% \loadcode{haskell}{Probni kod}{probni_kod}{paper/cde/test_code_01.hs}