\chapter{Uvod}

Cilj ovog rada je istražiti primjenu čistog funkcijskog programskog jezika Haskell na praktičnom primjeru izrade pametne kuharice koja koristi metode strojnog učenja i optimizacije kako bi korisniku preporučila potencijalno zanimljiv recept ili stvorila tjedni jelovnik koji maksimalno iskorištava sastojke.

Ideja je proizašla iz nezadovoljstva nakon višegodišnjeg korištenja objektno orijentiranih programskih jezika te pojavom elemenata funkcijskog programiranja u popularnim programskim jezicima poput JavaScripta, C\#a i Jave.

Današnje interakcije ljudi i različitih područja ljudskog djelovanja postaju sve složenije te važnost računarske znanosti i računarstva kao glavnog vezivnog tkiva i kanala za razmjenu i obradu informacija postaje sve veća.

Iz tog razloga pojavljuju se sve složeniji sustavi i primjene programskih rješenja nad kojima se mora brzo iterirati uz maksimalnu sigurnost i kvalitetu. Takva dinamika zahtjeva efektivne alate za upravljanje kompleksnošću i tu se objektno orijentirani pristup pokazuje sve manje poželjnim izborom, dok se funkcijski pristup čini sve pogodnijim.

Ovaj trend korištenja funkcijskih programskih jezika prati sve veći broj manjih firmi koje nemaju teret zastarjelog \foreign{eng}{legacy} koda, ali ne zaostaju niti veće firme poput FaceBook-a koji je razvio svoj sustav Sigma\cite{sigma} za borbu protiv neželjenih poruka \foreign{eng}{spam} u Haskellu.